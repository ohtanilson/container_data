\section{Consistency of our recovered data with the interviews}\label{sec:interview}

In this section, we provide interview-based evidence on the consistency of our recovered data of container freight rate with historical experience of industry experts, Akimitsu Ashida and Hiroyuki Sato.

\subsection{Akimitsu Ashida, an ex-chairperson of Mitsui O.S.K Lines}

Concerning the increase in freight rates on Atlantic and European routes in the late 1980s, Mr. Akimitsu Ashida, former chairman of Mitsui O.S.K. Lines (MOL) answered about the situation on liner routes. He served as the company's European Division Manager from 1985-86, and responded to our e-mail inquiry, on February 28, 2022.\\

\textcolor{blue}{\textbf{Are the recovered shipping price and quantity data reasonable benchmark?} \\
Ashida: Yes or no. Reason.
}

\textbf{Regarding Far East-Europe routes} \\
Ashida: Unlike Transpacific routes (where U.S. Customs publishes data), European routes did not have a system whereby customs authorities published statistics on container transport volumes. It was a significant difference from the Transpacific routes, and FEFC did not have the task of compiling and notifying its members of actual container transport volumes. Therefore, it is not easy to find the data even today. \textcolor{blue}{[Thus, the data looks reasonable?]}

\textbf{Freight Rate Increases in the Late 1980s} \\
Ashida: One of the reasons for this is that alliances were functioning on Far East-Europe routes. Freight rates were regularly rising. In addition, surcharges such as Bunker Adjustment Factor (BAF) and Currency Adjustment Factor (CAF) were collected without fail. Under these circumstances, the Plaza Accord of 1985 raised the yen-dollar exchange rate from 240 yen to about 140 yen. It has caused dollar-based freight rates to rise sharply, especially for cargo originating from Japan. Non-Japanese carriers earned even more profits because their yen-denominated costs were relatively small.  \textcolor{blue}{[Thus, the data looks reasonable?]}

\textbf{Why the conference Was Still Functioning on the Far East-Europe Routes}\\
Ashida: One reason is that there were no powerful non-conference member shipping lines. At the time, Japanese shippers tended to avoid unions for exports from Japan, which accounted for 50 per cent of all cargo shipped from the Far East, so competition was centered on allied shipping lines. The trend was primarily disrupted by the introduction large newly built vessels by non-conference members from 1990 onward.

Ashida: I also recall that about 35 per cent of cargo on the Far East-Europe routes was handled through forwarders, unlike the Transpacific routes. I also remember that the Far East-Europe routes were less active in discount negotiations than the Transpacific routes because the inland transport distances were shorter, and lower ocean freight rates would reduce the room for profit for the company. \textcolor{blue}{[Thus, the data looks reasonable?]}

\textbf{Internal mechanism within conference firms} \\
Ashida: A consortium was formed among conference member shipping companies, and among them, TRIO (two Japanese (NYK and MOL), one German (P \& O), and two British (Hapag-Lloyd and Ben Line)) held about 60 per cent of the cargo. The TRIO group had a perfect pooling system. P\&O developed the group's pooling system with the help of consultants. P\&O had a little over 30 per cent share, NYK and MOL had less than 30 per cent, and Hapag-Lloyd and Ben Line a little less than 40 per cent. The share remained unchanged from the inception of TRIO to its dissolution. Only the costs associated with loading/unloading the container vessels were deducted from the freight charges, and all the rest was deposited once into the pool. The collected CAF was also pooled. Each company reported the amount deposited in the pool monthly, and after one year, the excess or deficiency against each company's share was adjusted. MOL's actual results always exceeded its allotted share, and I recall that it paid about 1 to 1.5 billion yen as a settlement.

\textbf{The motives behind the creation of the shipping alliance}\\
Ashida: Around 1990, Japanese manufacturers lost competitiveness due to the strong yen and shifted their production bases to Asia, impacting container transport volumes. Japanese exports declined while Southeast Asian exports grew significantly, and Japanese shipping companies had a tiny share of TRIO's Asian exports. Moreover, the TRIO Group's system was relatively rigid and could not carry the increased cargo originating from Asia. It was one of the main reasons I later wanted MOL to leave TRIO and create a new alliance to work on Asian cargo.

\textcolor{blue}{
\textbf{Are the recovered industry-level newbuilding, secondhand, scrap price data reasonable benchmark?} \\
Ashida: Yes or no. Reason.
}

\subsection{Hiroyuki Sato, an ex-vice chairperson of Mitsui O.S.K Lines}

We interviewed Mr. Hiroyuki Sato, former vice president of MOL, about the Conference on the Transpacific routes and the situation on liner routes in the same period. He had been in charge of sales for European service routes since 1969 and had been in charge of North American service routes since 1974. In addition, he responded to our onsite interview on November 17, 2021 and e-mail inquiries  \textcolor{blue}{, on December 15, 2021 and February 28, 2022}.\\

\textcolor{blue}{\textbf{Are the recovered shipping price and quantity data reasonable benchmark?} \\
Sato: Yes or no. Reason.
}

\textbf{About the  Conference for the Transpacific routes} \\
Sato: There are two types of shipping alliances: closed conferences and open conferences. Typical of the former is the Far Eastern Freight Conference (FEFC), which had a strict screening process for membership. The number of voyages of individual members was clearly defined. Some members conducted consortia, such as the TRIO group, which introduced a strict pooling system. There were also closed conferences other than the Far-East Europe route (e.g., the Japan-India and Pakistan routes), where a pooling system also existed. Other pooling was executed for specific cargoes, such as wool and cotton. In the late '70s, the Conference's regular meetings were attended by managers of each company. There were also U.S. shipping companies and Maersk representatives. Decisions made during the Conference were forwarded to their head office. \textcolor{blue}{[Thus, the data looks reasonable?]}

\textbf{Determination of container freight rates} \\
Sato: In the 1970s, containerization was progressing on all routes, but freight rates were determined based on weight or measure the same as conventional vessels, rather than box rates, which were rates per container. There was also an arrangement called ``minimum revenue per container". In addition, a few percent discounts was applied to long-term contract shippers using a double freight contract with the shipper. On the Transpacific routes, members were free to come and go. Therefore even principal members, such as Sealand, often threatened to leave or dared to leave the Conference if they were dissatisfied with the other members' response. The Conference could operate competitive vessels at lower rates on the Far East-Europe routes to compete with the target non-conference carriers. On the Transpacific routes, however, it was not possible, and bargaining with non-conference shipping companies was considered necessary.

Sato: In the 1980s, there was a shift in BOX rates. The enactment of the Shipping Act of 1984 was a significant reason for this shift. The period after the enactment of the Shipping Act was when space charters were no longer viable, which led to the formation of alliances in the 1990s. \textcolor{blue}{[Thus, the data looks reasonable?]}


\textcolor{blue}{
\textbf{Are the recovered industry-level newbuilding, secondhand, scrap price data reasonable benchmark?} \\
Sato: Yes or no. Reason.
}